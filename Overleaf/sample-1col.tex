%% The first command in your LaTeX source must be the \documentclass command.
%%
%% Options:
%% twocolumn : Two column layout.
%% hf: enable header and footer.
\documentclass[
% twocolumn,
% hf,
]{ceurart}

%%
%% One can fix some overfulls
\sloppy

%%
%% Minted listings support 
%% Need pygment <http://pygments.org/> <http://pypi.python.org/pypi/Pygments>
\usepackage{listings}
%% auto break lines
\lstset{breaklines=true}

%%
%% end of the preamble, start of the body of the document source.
\begin{document}

%%
%% Rights management information.
%% CC-BY is default license.
\copyrightyear{2025}
\copyrightclause{Copyright for this paper by its authors.
  Use permitted under Creative Commons License Attribution 4.0
  International (CC BY 4.0).}

%%
%% This command is for the conference information
\conference{Woodstock'22: Symposium on the irreproducible science,
  June 07--11, 2022, Woodstock, NY}

%%
%% The "title" command
\title{Team Assigment - Recommender Systems Autumn 2024/25}

\tnotemark[1]
\tnotetext[1]{You can use this document as the template for preparing your
  publication. We recommend using the latest version of the ceurart style.}

%%
%% The "author" command and its associated commands are used to define
%% the authors and their affiliations.
\author[1]{Michael Gubler}[%
email=michael.gubler@stud.hslu.ch,
]
\author[1]{Eva-Maria Bonin}[%
email=eva-maria.bonin@stud.hslu.ch,
]
\author[1]{Jiaqi Yu}[%
email=jiaqi.yu@stud.hslu.ch,
]

\fnmark[1]
\address[1]{HSLU}

%% Footnotes
\cortext[1]{Corresponding author.}
\fntext[1]{These authors contributed equally.}

%%
%% The abstract is a short summary of the work to be presented in the
%% article.
\begin{abstract}
  A clear and well-documented \LaTeX{} document is presented as an
  article formatted for publication by CEUR-WS in a conference
  proceedings. Based on the ``ceurart'' document class, this article
  presents and explains many of the common variations, as well as many
  of the formatting elements an author may use in the preparation of
  the documentation of their work.
\end{abstract}

%%
%% Keywords. The author(s) should pick words that accurately describe
%% the work being presented. Separate the keywords with commas.
\begin{keywords}
  LaTeX class \sep
  paper template \sep
  paper formatting \sep
  CEUR-WS
\end{keywords}

%%
%% This command processes the author and affiliation and title
%% information and builds the first part of the formatted document.
\maketitle

\section{Introduction}



\begin{itemize}
    \item \textbf{Question 1} - How would you develop an algorithm to win the Kaggle competition?
    \item \textbf{Question 2} - What would you propose to solve Deezer's general recommendation problems?
    \item \textbf{Question 3} - Do the two solutions overlap, in what way and why or why not?
\end{itemize}


\section{Teamwork}

MG [please add] - Set up of collaborative tools (github), data cleaning and preprocessing, EDA, model: user-based collaborative filtering (incl. write-up of results)
EMB - Set up of Overleaf document, draft responses to questions incl. development of real world system, obtaining additional contextual data from Deezer and public datasets, model: item-based collaborative filtering (incl. write-up of results)
JY - Normalising scale variables, visual EDA, model: matrix factorisation (incl. write-up of results)
All - Final model; final edits to document.



\section{Question 1: Kaggle competition}

\subsection{Competition goal and data}

The goal of the Kaggle competition is to maximize predictive accuracy with regard to whether the next recommended song was listened to for 30 seconds or more.

The data are provided already split into train and test sets. The training data contains XXX rows, while the test data contains YYY rows. The following features are available:

\begin{itemize}
    \item \textbf{media\_id} - ID of the song listened by the user
    \item \textbf{album\_id} - ID of the album of the song
    \item \textbf{media\_duration} - Duration of the song
    \item \textbf{user\_gender} - Gender of the user
    \item \textbf{user\_id} - Annonymised id of the user
    \item \textbf{context\_type} - Type of content where the song was listened: playlist, album ...
    \item \textbf{release\_date} - Release date of the song with the format YYYYMMDD
    \item \textbf{ts\_listen} - Timestamp of the listening in UNIX time
    \item \textbf{platform\_name} - Type of OS
    \item \textbf{platform\_family} - Type of device
    \item \textbf{user\_age} - Age of the user
    \item \textbf{listen\_type} - If the song was listened in a flow or not
    \item \textbf{artist\_id} - ID of the artist of the song
    \item \textbf{genre\_id} - ID of the genre of the song
    \item \textbf{is\_listened} - Binary feature (1 = track was listened to for more than 30 seconds)
\end{itemize}


\subsubsection{Exploratory data analysis}
Some EDA here

\subsubsection{Data preprocessing}
From the above EDA, we can see that the following preprocessing steps are required:

\begin{itemize}
    \item \textbf{Missing values} - Given the small number (??? check EDA) of missing values, rows with missing data were deleted.
    \item \textbf{Data types} - Ensure correct data types, especially for date/time variables.
    \item \textbf{Encode categorical variable} - Convert categorical data to numeric if necessary
    \item \textbf{Normalise or scale numeric features} - Only necessary if our models are sensitive to scale, but I expect it will be necessary
    \item \textbf{Feature engineering} - Extract features such as [years from dates, user age group; what will be required to group users later?; timestamp: maybe day vs night?] and generate feature for optimisation (user listened to song for 30 seconds or more)
\end{itemize}

\subsubsection{Feature engineering}

We extracted the following new features from the data ...



\subsection{Modelling approaches}


Use of external data: https://www.theserverside.com/feature/How-Pandora-built-a-better-recommendation-engine
- social tagging
- keyword annotations from music related websites using text processing techniques
- accustic and musical features --> Spotify. Does Deezer provide this as well?

\href{https://towardsdatascience.com/the-abc-of-building-a-music-recommender-system-part-i-230e99da9cad}{https://towardsdatascience.com/the-abc-of-building-a-music-recommender-system-part-i-230e99da9cad}

\href{https://www.kaggle.com/varshita/content-based-recommender-system}{https://www.kaggle.com/varshita/content-based-recommender-system}

\href{http://staff.ustc.edu.cn/~hexn/slides/sigir20-tutorial-CRS-slides.pdf}{http://staff.ustc.edu.cn/\~hexn/slides/sigir20-tutorial-CRS-slides.pdf} 



Collaborative filtering
\begin{itemize}
    \item User-based CF: Users with similar behaviours
    \item Item-based CF: Tracks frequently listened to together
    \item Matrix factorisation: Identify latent music preferences
\end{itemize}

\subsection{Evaluation metrics / model performance}

Precision@K
AUC
NDCG

\section{Question 2: Deezer real world system}

\subsection{The Deezer business challenge}

While accuracy of prediction is vital, Deezer has several additional business goals: It needs to attract new users, ensure retention of existing users, and to ensure satisfaction in the service. The latter directly impacts user retention, and will indirectly impact new sign-ups based on reputation. 

A real-world recommender system to incorporate the following considerations to meet these goals:

\begin{itemize}
    \item \textbf{Cold start} - The model needs to cater to new users, and ensure new songs are served, but both lack a listening history.
    \item \textbf{Context awareness} - Recommendations need to be adapted to time of day, user activity, user mood, and other contextual actors.
    \item \textbf{Change in preferences over time} - Recommendations need to evolve along with user preferences.
    \item \textbf{Diversity and discovery} - Recommendations should not be too repetitive, and should balance popular results with new songs.
    \item \textbf{Scalability} - The model needs to be able to handle lange amounts of sequential data.
\end{itemize}

\subsection{Proposed system}

While collaborative filtering remains an important part of the recommender system, this approach on its own will fail for cold starts (both users and songs). 

\textit{Content-based filtering} can be employed to address the cold start problem. Recommendations can utilise meta data such as artist, genre. Natural language processing could be used to analyse lyrics and provide further information about the song. Deep learning methods (audio embeddings) can be used to find songs with similar acoustic features. While this approach 

\textit{Context-aware recommendations}
\textit{Reinforcement learning}

Dealing with multiple account users --> Zalando presentation
Transformers: Better understand sequential nature of user actions based on current context (https://research.google/blog/transformers-in-music-recommendation/)



\section{Question 3: Comparison}



\begin{table}
  \caption{Differences between Kaggle and Deezer models}
  \label{tab:differences}
  \begin{tabular}{lll}
    \toprule
    Issue & Kaggle model & Deezer model\\
    \midrule
    \texttt{Goal} & Maximise prediction accuracy& Maximise user engagement/retention and music diversity \\
    \texttt{Data}& Static tabular dataset & Real time data stream\\
    \texttt{Cold start}& Will struggle with new users and songs& Addressed by content-based filtering\\
    \texttt{Adaptability}& Static recommendations & Should learn and adjust\\
    \texttt{Diversity}& Focus on accuracy over diversity & Should ensure discovery\\
    \texttt{Evaluation metrics}& Precision@K, AUC, NDCG & Engagement, retention, diversity\\
    \texttt{Model type}& Collaborative fildering & Hybrid\\
    \bottomrule
  \end{tabular}
\end{table}



\section{Sectioning Commands}

Your work should use standard \LaTeX{} sectioning commands:
\verb|\section|, \verb|\subsection|,
\verb|\subsubsection|, and
\verb|\paragraph|. They should be numbered; do not remove
the numbering from the commands.

Simulating a sectioning command by setting the first word or words of
a paragraph in boldface or italicized text is not allowed.

\section{Tables}

The ``\verb|ceurart|'' document class includes the ``\verb|booktabs|''
package --- \url{https://ctan.org/pkg/booktabs} --- for preparing
high-quality tables.

Table captions are placed \textit{above} the table.

Because tables cannot be split across pages, the best placement for
them is typically the top of the page nearest their initial cite.  To
ensure this proper ``floating'' placement of tables, use the
environment \verb|table| to enclose the table's contents and the
table caption. The contents of the table itself must go in the
\verb|tabular| environment, to be aligned properly in rows and
columns, with the desired horizontal and vertical rules.

Immediately following this sentence is the point at which
Table~\ref{tab:freq} is included in the input file; compare the
placement of the table here with the table in the printed output of
this document.

\begin{table*}
  \caption{Frequency of Special Characters}
  \label{tab:freq}
  \begin{tabular}{ccl}
    \toprule
    Non-English or Math&Frequency&Comments\\
    \midrule
    \O & 1 in 1,000& For Swedish names\\
    $\pi$ & 1 in 5& Common in math\\
    \$ & 4 in 5 & Used in business\\
    $\Psi^2_1$ & 1 in 40,000& Unexplained usage\\
  \bottomrule
\end{tabular}
\end{table*}

To set a wider table, which takes up the whole width of the page's
live area, use the environment \verb|table*| to enclose the table's
contents and the table caption.  As with a single-column table, this
wide table will ``float'' to a location deemed more
desirable. Immediately following this sentence is the point at which
Table~\ref{tab:commands} is included in the input file; again, it is
instructive to compare the placement of the table here with the table
in the printed output of this document.

\begin{table}
  \caption{Some Typical Commands}
  \label{tab:commands}
  \begin{tabular}{ccl}
    \toprule
    Command &A Number & Comments\\
    \midrule
    \texttt{{\char'134}author} & 100& Author \\
    \texttt{{\char'134}table}& 300 & For tables\\
    \texttt{{\char'134}table*}& 400& For wider tables\\
    \bottomrule
  \end{tabular}
\end{table}


\section{Figures}

The ``\verb|figure|'' environment should be used for figures. One or
more images can be placed within a figure. If your figure contains
third-party material, you must clearly identify it as such, as shown
in the example below.
\begin{figure}

\end{figure}

Your figures should contain a caption which describes the figure to
the reader. Figure captions go below the figure. Your figures should
also include a description suitable for screen readers, to
assist the visually-challenged to better understand your work.

Figure captions are placed below the figure.




\section*{Declaration on Generative AI}

  
 \noindent{\em Or (by using the activity taxonomy in ceur-ws.org/genai-tax.html):\newline}
 During the preparation of this work, the author(s) used X-GPT-4 and Gramby in order to: Grammar and spelling check. Further, the author(s) used X-AI-IMG for figures 3 and 4 in order to: Generate images. After using these tool(s)/service(s), the author(s) reviewed and edited the content as needed and take(s) full responsibility for the publication’s content. 


%%
%% If your work has an appendix, this is the place to put it.
\appendix

\section{Bibliography}
\begin{lstlisting}
\bibliography{recommender_systems}
\end{lstlisting}

\end{document}

%%
%% End of file
